%%%%%%%%%%%%%%%%%%%%%%%%%%%%%%%%%%%%%%%%%
% "ModernCV" CV and Cover Letter
% LaTeX Template
% Version 1.11 (19/6/14)
%
% This template has been downloaded from:
% http://www.LaTeXTemplates.com
%
% Original author:
% Xavier Danaux (xdanaux@gmail.com)
%
% License:
% CC BY-NC-SA 3.0 (http://creativecommons.org/licenses/by-nc-sa/3.0/)
%
% Important note:
% This template requires the moderncv.cls and .sty files to be in the same 
% directory as this .tex file. These files provide the resume style and themes 
% used for structuring the document.
%
%%%%%%%%%%%%%%%%%%%%%%%%%%%%%%%%%%%%%%%%%

%----------------------------------------------------------------------------------------
%	PACKAGES AND OTHER DOCUMENT CONFIGURATIONS
%----------------------------------------------------------------------------------------

\documentclass[11pt,a4paper,sans]{moderncv} % Font sizes: 10, 11, or 12; paper sizes: a4paper, letterpaper, a5paper, legalpaper, executivepaper or landscape; font families: sans or roman

\usepackage[utf8]{inputenc}

\moderncvstyle{casual} % CV theme - options include: 'casual' (default), 'classic', 'oldstyle' and 'banking'
\moderncvcolor{blue} % CV color - options include: 'blue' (default), 'orange', 'green', 'red', 'purple', 'grey' and 'black'

\usepackage[scale=0.80]{geometry} % Reduce document margins
%\setlength{\hintscolumnwidth}{3cm} % Uncomment to change the width of the dates column
%\setlength{\makecvtitlenamewidth}{10cm} % For the 'classic' style, uncomment to adjust the width of the space allocated to your name

%----------------------------------------------------------------------------------------
%	NAME AND CONTACT INFORMATION SECTION
%----------------------------------------------------------------------------------------

\firstname{Fernando} % Your first name
\familyname{Pujaico Rivera} % Your last name

% All information in this block is optional, comment out any lines you don't need
\title{Curriculum Vitae}
%\address{Rua Manoel Antunes Novo 1000}{Campinas, SP, CEP:13.084-175}
\mobile{(19) 992612067}
%\phone{(000) 111 1112}
%\fax{(000) 111 1113}
\email{fpujaico@decom.fee.unicamp.br}
%\homepage{http://trucomanx.users.sourceforge.net/}{http://trucomanx.users.sourceforge.net/} % The first argument is the url for the clickable link, the second argument is the url displayed in the template - this allows special characters to be displayed such as the tilde in this example
%%\extrainfo{RNE: V566622-O, CPF: 233.534.528-18}

\photo[70pt][0.4pt]{pictures/foto3.eps} % The first bracket is the picture height, the second is the thickness of the frame around the picture (0pt for no frame)
%\quote{"A witty and playful quotation" - John Smith}

%----------------------------------------------------------------------------------------

\begin{document}

\makecvtitle % Print the CV title


%----------------------------------------------------------------------------------------
%	DATOS
%----------------------------------------------------------------------------------------

\section{Dados Pessoais}
\cvitem{Nascimento}{17 Dezembro de 1982}
\cvitem{Naturalidade}{Peru}
\cvitem{Endereço}{Rua Geraldo Vitorino 188, Jardim América, Lavras, MG, Brasil, CEP:37200-000}
\cvitem{Celular}{(19) 992612067}
\cvitem{e-mail}{fpujaico@decom.fee.unicamp.br}
\cvitem{RNE}{V566622-O}
\cvitem{CPF}{233.534.528-18}
\cvitem{Currículo Lattes}{http://lattes.cnpq.br/1562723678793624}


%----------------------------------------------------------------------------------------
%	EDUCATION SECTION
%----------------------------------------------------------------------------------------

\section{Títulos Obtidos}

\cventry{2014}{Doutorado em Engenharia Elétrica}
	      {Universidade Estadual de Campinas}
	      {UNICAMP}{Brasil}
	      {Título: Algoritmos Bit-Flipping Para Decodificação Conjunta de Fontes Correlacionadas Em Canais Ruidosos.}
\cventry{2011}{Mestrado em Engenharia Elétrica}
	      {Universidade Estadual de Campinas}
	      {UNICAMP}{Brasil}
	      {Título: Algoritmos de Decodificação Abrupta para Códigos LDGM.}  % Arguments not required can be left empty
\cventry{2008}{Engenheiro Eletrônico}
	      {Universidade Nacional de Engenharia}
	      {UNI}{Peru}
	      {Título: ``Tomógrafo de Resistividad Eléctrica Aplicado al Estudio del Crecimiento de las Raíces''.}
\cventry{2006}{Bacharel em Ciências com Menção em Engenharia Eletrônica}
	      {Universidade Nacional de Engenharia}
	      {UNI}{Peru}{}

\subsection{Áreas de Atuação}

\cvitem{}{Engenharia Eletrônica, Teoria da informação, Códigos corretores de erro, Programação, Desenho eletrônico, Processamento digital de sinais.}

%\section{Masters Thesis}
%
%\cvitem{Title}{\emph{Money Is The Root Of All Evil -- Or Is It?}}
%\cvitem{Supervisors}{Professor James Smith \& Associate Professor Jane Smith}
%\cvitem{Description}{This thesis explored the idea that money has been the cause of untold anguish and suffering in the world. I found that it has, in fact, not.}

%----------------------------------------------------------------------------------------
%	WORK EXPERIENCE SECTION
%----------------------------------------------------------------------------------------

\section{Experiência}

\subsection{Experiência Docente}

%\cventry{2012--Present}{}{}{}{}	{ \newline{} }

\cventry{Novembro 2016}{Minicurso: Speckle Laser Dinâmico em Biosistemas}
	      {Faculdade de Engenharia Agrícola}
	      {Universidade Estadual de Campinas}{Brasil}
	      {8 horas }
	      

\cventry{2do Semestre 2013}{Estágio de Capacitação Docente: PED C}
			  {GL100 }{Matemática I}{}
			  {Entidade: FCA UNICAMP }

\cventry{1ro Semestre 2010}{Estágio de Capacitação Docente: PED C}
			  {EE881 }{Princípios de Comunicações}{}
			  {Entidade: FEEC UNICAMP }

\cventry{2008}{Professor}
	      {Linguagem C++ }{ Nível I}{}
	      {Entidade: Centro Cultural de Engenharia Elétrica ``Santiago 
	      Antúnez de Mayolo'' (CCIESAM) da Faculdade de Engenharia Elétrica 
	      e Eletrônica da Universidade Nacional de Engenharia. Peru.}

\subsection{Experiência Profissional}


\cventry{2015 -- 2019}{Pós-Doutorado}
	      {Universidade Federal de Lavras}
	      {UFLA}{Brasil}
	      {Departamento Engenharia / Centro de Desenvolvimento de Instrumentação Aplicada à Agropecuária. }


\cventry{2007 -- 2008}{Pesquisador}
		      {Instituto de Pesquisa e Desenvolvimento da Faculdade 
		      de Engenharia Civil}{Universidade Nacional de Engenharia }{Peru}
		      {Tipo de contratação: Laboral\newline{}
		      Descrição: Desenho, Construção e Processamento de Dados de um 
		      Acelerômetro Para a Rede Nacional de Acelerômetros do CISMID - II.}

\cventry{2006 -- 2008}{Pesquisador}
		      {Instituto de Pesquisa e Desenvolvimento da Faculdade 
		      de Engenharia Civil}{Universidade Nacional de Engenharia }{Peru}
		      {Tipo de contratação: Laboral\newline{}
		      Descrição: Desenho e  Construção  de  um  Sistema  de  
		      Adquisição  de  Dados para  um  Ensaio  Dinâmico  de  Pilotes.}

\cventry{2005 -- 2006}{Pesquisador}
		      {Centro de Pesquisa e Desenvolvimento da Faculdade de 
		      Engenharia Eléctrica e Eletrônica}{Universidade Nacional de Engenharia }{Peru}
		      {Tipo de contratação: Laboral\newline{}
		      Descrição: Estudo, Avaliação, Desenho e implementação de um sistema Bioeletrônico II.}

\cventry{2004 -- 2006}{Pesquisador}
		      {``International Potato Center''}{Lima}{Peru}
		      {Tipo de contratação: Bolsista\newline{}
		      Descrição: Construção  de  um  Tomógrafo  de  Resistividade 
		      Elétrica  aplicado ao  estudo  do  crescimento  das  raízes.}
%----------------------------------------------------------------------------------------
%	AWARDS SECTION
%----------------------------------------------------------------------------------------
%\section{Awards}
%\cvitem{2011}{School of Business Postgraduate Scholarship}
%\cvitem{2010}{Top Achiever Award -- Commerce}


\section{Publicação de Trabalhos}
\subsection{Livros}
\cventry{2016}{A practical guide to biospeckle laser analysis: theory and software}
	      {ISBN: 9-788581-270517}{2016}{Ed. UFLA}
	      {http://repositorio.ufla.br/jspui/handle/1/12119}

\subsection{Capítulos de Livros}
\cventry{2019}{Engenharias, ciência e tecnologia 4}
	      {ISBN: 9788572470872}{2019}{Editora Atena}
	      {DOI:10.22533/at.ed.872193101}

\subsection{Artigos Publicados em Revistas}
\cventry{2019}{Computers and Electronics in Agriculture}
	      {DOI: 10.1016/j.compag.2019.01.051}{}{}
	      {Artigo: ``Sound as a qualitative index of speckle laser to monitor biological systems''.}

\cventry{2018}{Theoretical and Applied Engineering}
	      {DOI: 10.31422/taae.v2i2.5}{}{}
	      {Artigo: ``The use of particle image velocimetry for displacement measurements in steel columns subjected to buckling''.}
	      
\cventry{2018}{Optics and Laser Technology}
	      {DOI: 10.1016/j.optlastec.2018.07.006}{}{}
	      {Artigo: ``Diode laser reliability in dynamic laser speckle application: Stability and signal to noise ratio''.}
	      
\cventry{2018}{Journal of Food Measurement and Characterization}
	      {DOI: 10.1007/s11694-018-9839-8}{}{}
	      {Artigo: ``Measurement of water activities of foods at different temperatures using biospeckle laser''.}

\cventry{2018}{Engenharia Agrícola, ISSN:0100-6916}
	      {DOI: 10.1590/1809-4430-eng.agric.v38n2p159-165/2018}{}{}
	      {Artigo: ``Analysis of elasticity in woods submitted to the static bending test using the particle image velocimetry (PIV) technique''.}

\cventry{2017}{Journal of Biomedical Optics}
	      {DOI: 10.1117/1.JBO.22.4.045010}{}{}
	      {Artigo: ``Dynamic laser speckle analyzed considering inhomogeneities in the biological sample''.}
	      
\cventry{2017}{Optics Communications}
	      {DOI: 10.1016/j.optcom.2017.03.015}{}{}
	      {Artigo: ``Selection of statistical indices in the biospeckle laser analysis regarding filtering actions''.}
	      
\cventry{2014}{IEEE Communications Letters}
	      {DOI: 10.1109/LCOMM.2014.2377237}{}{}
	      {Artigo: ``Optimal  Rate for Joint Source-Channel Coding of Correlated Sources Over Orthogonal Channels''.}

\subsection{Artigos Publicados em Anais de Eventos}

\cventry{2015}{I Congresso Mineiro de Engenharia e Tecnologia}
	      {Brasil}{}{http://www.eventos.ufla.br/comet/anais/ANAIS\_COMET\_2015\_1ed\_FINAL.pdf}
	      {Artigo: Diferenciação da Crosta Superficial do Solo por Meio de Técnicas Óticas}

\cventry{2013}{XXXI Simpósio Brasileiro de Telecomunicações}
	      {Brasil}{DOI: 10.14209/sbrt.2013.95}{http://gestao.sbrt.org.br/simposios/artigo/visualizar/a/145}
	      {Artigo: Algoritmo Para Decodificação e Fusão De Dados Correlacionados Em Redes De Sensores Sem Fio.}

\cventry{2012}{XXX Simpósio Brasileiro de Telecomunicações}
	      {Brasil}{}{http://gestao.sbrt.org.br/simposios/artigo/visualizar/a/432}
	      {Artigo: Algoritmos de Decodificação Abrupta para Códigos LDGM.}

\cventry{2011}{XXIX Simpósio Brasileiro de Telecomunicações}
	      {Brasil}{}{www.sbrt.org.br/sbrt2011/progtec.pdf}
	      {Artigo: Decodificação Iterativa Conjunta Fonte-Canal.}

\cventry{2007}{``XVII  Congreso Nacional de Ingeniería, Mecánica, Eléctrica y Ramas Afines''}
	      {Peru}{}{}
	      {Artigo: ``Tomógrafo de Resistividad Eléctrica Aplicado al Estudio del Crecimiento de los Tubérculos de la Papa''.}


%----------------------------------------------------------------------------------------
%	COMMUNICATION SKILLS SECTION
%----------------------------------------------------------------------------------------

\section{Apresentações}

\cventry{2013}{Algoritmo Para Decodificação e Fusão De Dados Correlacionados Em Redes De Sensores Sem Fio}
	      {}{}{}
	      {XXXI Simpósio Brasileiro de Telecomunicações, Brasil}

\cventry{2012}{Algoritmos de Decodificação Abrupta para Códigos LDGM}
	      {}{}{}
	      {XXX Simpósio Brasileiro de Telecomunicações, Brasil}

\cventry{2011}{Decodificação Iterativa Conjunta Fonte-Canal}
	      {}{}{}
	      {XXIX Simpósio Brasileiro de Telecomunicações, Brasil}

\cventry{2008}{Ferramentas Livres para Desenho CAD, KiCad}
	      {}{}{}
	      {``I Jornada de Software Libre GNU/LINUX 2008 FIEE-UNI'', Peru}

\cventry{2008}{O Compilador PIC-GCC e as Bibliotecas PIC-GCC-Library}
	      {}{}{}
	      {``I Jornada de Software Libre GNU/LINUX 2008 FIEE-UNI'', Peru}

\cventry{2007}{``Tomógrafo de Resistividad Eléctrica Aplicado al Estudio del Crecimiento de los Tubérculos de la Papa''}
	      {}{}{}
	      {``XVII  Congreso Nacional de Ingeniería, Mecánica, Eléctrica y Ramas Afines'', Peru}

\cvitem{2006}{Expositor dos Projetos da Área de Processamento de Sinais e Sistemas 
	      do Centro de Investigação e Desenvolvimento da Faculdade de Engenharia
	      Elétrica e Eletrônica, ``Feria Científica Tecnológica y Empresarial UNI 2006'', Peru}

	       
%----------------------------------------------------------------------------------------
%	COMPUTER SKILLS SECTION
%----------------------------------------------------------------------------------------
\section{Orientador}
\subsection{Coorientador}
\cventry{2016}{Desenvolvimento de uma técnica óptica para caracterização da presença de crosta superficial do solo}
			{Barreto, Bianca Batista}
			{Mestrado em Engenharia Agrícola}{Universidade Federal de Lavras}
			{http://repositorio.ufla.br/jspui/handle/1/11903}

\section{Participação em Bancas de Trabalhos de Conclusão}
\subsection{Doutorado}
\cventry{2016}{Digitalização de Deformações Físicas do Solo por Meio de uma Câmera  Digital}
			{Participação em banca de Diego Eduardo Costa Coelho}
			{Defesa de tese do doutorado do programa de Pós-Graduação em Engenharia Agrícola}{}
			{Universidade Federal de Lavras. Portaria CPGSS/PRPG Nro 987/2016 de 23/11/2016.}
\subsection{Mestrado}
\cventry{2015}{Influencia  da Intensidade do Laser nos Mapas de Atividade do Biospeckle}
			{Participação em banca de Renan Oliveira Reis}
			{Defesa de dissertação do programa de Pós-Graduação em Engenharia de Sistemas e Automação}{}
			{Universidade Federal de Lavras. Portaria CPGSS/PRPG Nro 655/2015 de 13/07/2015.}
\subsection{Qualificação de Doutorado}
\cventry{2016}{Participação na comissão avaliadora de Rodrigo Allan Pereira}
			{}
			{Exame de qualificação do programa de Pós-Graduação em Engenharia Agrícola}{}
			{Universidade Federal de Lavras.}
			
\subsection{Qualificação de Mestrado}
\cventry{2016}{Participação na comissão avaliadora de Eduardo Zampieri Ribeiro}
			{}
			{Exame de qualificação do programa de Pós-Graduação em Engenharia de Sistemas e Automação}{}
			{Universidade Federal de Lavras.}
%----------------------------------------------------------------------------------------
%	COMPUTER SKILLS SECTION
%----------------------------------------------------------------------------------------

\section{Projetos em Software Livre}

\cventry{2015 -- Atual}{Bio-Speckle Laser Tool Library}
			{}
			{}{}
			{Este pacote é um conjunto de funções, escritas em M-código, para o 
			processamento digital de imagens provenientes de um analises bio-speckle.
			A biblioteca deve ser usada em OCTAVE ou MATLAB.
			Podem ser achadas funções para o cálculo de: 
			Matriz de co-ocorrência,THSP, AVD, momento de inercia,
			Fujii, GD, PTD, etc.}

\cventry{2015 -- Atual}{PDS-IT Package}
			{\url{http://trucomanx.github.io/pdsit-pkg}}
			{}{}
			{Este pacote é um conjunto de funções, escritas em M-código, para trabalhar 
			com processamento digital de sinais e teoria da 
			informação em OCTAVE/MATLAB. Podem ser achadas 
			funções para: entropia de fontes binárias, 
			entropia conjunta de fontes binárias, 
			taxa de erro de bit no problema CEO. }
			
\cventry{2014 -- Atual}{PDS Project Library in Java}
			{\url{http://pdsplibj.sourceforge.net/}}
			{}{}
			{Conjunto de bibliotecas, escritas em linguagem Java, para 
			o processamento digital de sinais. Podem ser achadas 
			bibliotecas para: variáveis aleatórias, 
			vetores, matrizes, filtros 
			digitais, fontes digitais, velocimetria por imagem de partículas, etc.}

\cventry{2014 -- Atual}{LDPC Tools}
			{\url{https://launchpad.net/ldpc-tools}}
			{}{}
			{Conjunto de programas, escritos em linguagem C, para 
			trabalhar com matrizes de verificação de paridade 
			de baixa densidade do tipo: LDPC, EG-LDPC, LDGM, etc.}
			

\cventry{2011 -- Atual}{PDS Project Library}
			{\url{http://www.nongnu.org/pdsplibrary/}}
			{}{}
			{Conjunto de bibliotecas, escritas em linguagem C, para 
			o processamento digital de sinais. Podem ser achadas 
			bibliotecas para: variáveis aleatórias, números complexos, 
			vetores, matrizes, transformada de Fourier, filtros 
			digitais, fontes digitais, redes neuronais, etc.}

\cventry{2008 -- Atual}{PIC-GCC Library}
			{\url{http://pic-gcc-library.sourceforge.net/}}
			{}{}
			{Este projeto implementa  uma biblioteca estândar e de 
			dispositivos para o Compilador de C ``PIC-GCC'' de 
			microcontroladores PIC da família 16F de Microchip}

\cventry{2007 -- Atual}{Linux Communication}
			{\url{http://lnxcomm.sourceforge.net/}}
			{}{}
			{Uma biblioteca escrita em linguagem C para a comunicação 
			mediante a porta serial}

%----------------------------------------------------------------------------------------
%	LANGUAGES SECTION
%----------------------------------------------------------------------------------------

\section{Idiomas}

%\cvitemwithcomment{Espanhol}{Língua Materna}{}
\cvitem{Espanhol}{Língua Materna}
\cvitem{Português}{Lê Bem, Escreve Bem, Compreende Bem, Fala Bem}
\cvitem{Inglês}{Lê Bem, Escreve Razoavelmente, Compreende pouco, Fala pouco}

 
%----------------------------------------------------------------------------------------
%	COMPUTER 
%----------------------------------------------------------------------------------------

\section{Linguagens de Computador}

\cvitem{C}{Linguagem C - Nível avançado }
\cvitem{M-código}{Linguagem MATLAB - Nível intermediário}
\cvitem{C++}{Linguagem C++ - Nível intermediário}
\cvitem{Java}{Linguagem Java - Nível intermediário}
\cvitem{LaTeX}{Linguagem LaTex - Nível intermediário}
\cvitem{Java/Android}{Desenvolvimento de Aplicativos Android - Nível básico}

%\section{Formação Complementar}

%%\subsection{ Cursos de Formação Complementar }
%\cventry{2008}{Automatização por PLC}
%	      {30 Horas}
%	      {}{}
%	      {Centro Cultural de Engenharia Elétrica ``Santiago Antúnez de Mayolo'' 
%	      (CCIESAM) da Faculdade de Engenharia Elétrica e Eletrônica da Universidade 
%	      Nacional de Engenharia, Peru.}

%\cventry{2007}{Compatibilidade Electromagnética}
%	      {9 Horas}
%	      {}{}
%	      {Secção de Pós-Graduação e Segunda Especialização da Faculdade de 
%	      Engenharia Elétrica e Eletrônica da Universidade Nacional de 
%	      Engenharia, Peru.}


%\cventry{2004}{Ensamblagem de Computadoras}
%	      {20 horas}
%	      {}{}
%	      {Centro de Capacitação Administrativa da Universidade Nacional de 
%	      Engenharia, a través do Centro de Computo ARUNI, Peru.}

%\cventry{2003}{Borland C++ 5.5 Nivel II}
%	      {20 horas}
%	      {}{}
%	      {Centro de Extensão e Projeção Social da Universidade Nacional de Engenharia.}

%\cventry{2003}{Programação em PHP}
%	      {24 horas}
%	      {}{}
%	      {Rama Estudantil IEEE da Universidade Nacional de Engenharia, Peru.}

%\cventry{2001}{Borland C++ Nivel I}
%	      {25 horas}
%	      {}{}
%	      {Centro de Especialização em Rede e Telemática da Faculdade de 
%	      Engenharia Elétrica e Eletrônica da Universidade Nacional de 
%	      Engenharia, Peru.}

%----------------------------------------------------------------------------------------
%	INTERESTS SECTION
%----------------------------------------------------------------------------------------

\section{Interesses}

\renewcommand{\listitemsymbol}{-~} % Changes the symbol used for lists
\cvlistdoubleitem{Fotografia}{Correr}
\cvlistdoubleitem{Ocarina}{Programação em C}
\cvlistdoubleitem{Energias Renováveis}{Raw Food}
%\cvlistdoubleitem{}{}

\end{document}
