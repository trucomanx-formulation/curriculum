%%%%%%%%%%%%%%%%%%%%%%%%%%%%%%%%%%%%%%%%%
% "ModernCV" CV and Cover Letter
% LaTeX Template
% Version 1.11 (19/6/14)
%
% This template has been downloaded from:
% http://www.LaTeXTemplates.com
%
% Original author:
% Xavier Danaux (xdanaux@gmail.com)
%
% License:
% CC BY-NC-SA 3.0 (http://creativecommons.org/licenses/by-nc-sa/3.0/)
%
% Important note:
% This template requires the moderncv.cls and .sty files to be in the same 
% directory as this .tex file. These files provide the resume style and themes 
% used for structuring the document.
%
%%%%%%%%%%%%%%%%%%%%%%%%%%%%%%%%%%%%%%%%%

%----------------------------------------------------------------------------------------
%	PACKAGES AND OTHER DOCUMENT CONFIGURATIONS
%----------------------------------------------------------------------------------------

\documentclass[11pt,a4paper,sans]{moderncv} % Font sizes: 10, 11, or 12; paper sizes: a4paper, letterpaper, a5paper, legalpaper, executivepaper or landscape; font families: sans or roman

\usepackage[utf8]{inputenc}

\moderncvstyle{casual} % CV theme - options include: 'casual' (default), 'classic', 'oldstyle' and 'banking'
\moderncvcolor{blue} % CV color - options include: 'blue' (default), 'orange', 'green', 'red', 'purple', 'grey' and 'black'

\usepackage[scale=0.80]{geometry} % Reduce document margins
%\setlength{\hintscolumnwidth}{3cm} % Uncomment to change the width of the dates column
%\setlength{\makecvtitlenamewidth}{10cm} % For the 'classic' style, uncomment to adjust the width of the space allocated to your name

%----------------------------------------------------------------------------------------
%	NAME AND CONTACT INFORMATION SECTION
%----------------------------------------------------------------------------------------

\firstname{Fernando} % Your first name
\familyname{Pujaico Rivera} % Your last name

% All information in this block is optional, comment out any lines you don't need
\title{Curriculum Vitae}
%\address{Rua Manoel Antunes Novo 1000}{Campinas, SP, CEP:13.084-175}
\mobile{+55 (35) 984071422}
%\phone{(000) 111 1112}
%\fax{(000) 111 1113}
\email{fernando.pujaico.rivera@gmail.com}
%\homepage{http://trucomanx.users.sourceforge.net/}{http://trucomanx.users.sourceforge.net/} % The first argument is the url for the clickable link, the second argument is the url displayed in the template - this allows special characters to be displayed such as the tilde in this example
%%\extrainfo{RNE: V566622-O, CPF: 233.534.528-18}

\photo[70pt][0.4pt]{pictures/foto3.png} % The first bracket is the picture height, the second is the thickness of the frame around the picture (0pt for no frame)
%\quote{"A witty and playful quotation" - John Smith}

%----------------------------------------------------------------------------------------

\begin{document}

\makecvtitle % Print the CV title


%----------------------------------------------------------------------------------------
%	DATOS
%----------------------------------------------------------------------------------------

\section{Personal information}
\cvitem{Born}{Peru - 17 December 1982}
\cvitem{Address}{Rua Barbosa Lima 638, Centro, Lavras, MG, Brazil, CEP:37200-000}
\cvitem{Cellphone}{+55 (35) 984071422}
\cvitem{E-mail}{fernando.pujaico.rivera@gmail.com}%{201518201@posgrad.ufla.br}
\cvitem{RNE}{V566622-O}
\cvitem{CPF}{233.534.528-18}
\cvitem{Curriculum Lattes}{\url{http://lattes.cnpq.br/1562723678793624}}
\cvitem{Orcid}{\url{https://orcid.org/0000-0002-4970-2818}}
\cvitem{Google Scholar}{\url{https://scholar.google.com/citations?user=wijGLBIAAAAJ}}
\cvitem{Web of Science ResearcherID}{\href{https://publons.com/researcher/3850161/fernando-pujaico-rivera/}{AAW-9842-2020}}

%----------------------------------------------------------------------------------------
%	EDUCATION SECTION
%----------------------------------------------------------------------------------------

\section{Education}

\cventry{2014}{PhD in Electrical Engineering}
	      {State University of Campinas (UNICAMP)}
	      {}{Brazil}
	      {Title: Bit-Flipping algorithms  for joint decoding of correlated sources in noisy channels.}
\cventry{2011}{Master's degree in Electrical Engineering}
	      {UNICAMP}
	      {}{Brazil}
	      {Title: Hard-decision decoding algorithms for LDGM codes.}  % Arguments not required can be left empty
\cventry{2008}{Electronic Engineer}
	      {National University of Engineering (UNI)}
	      {}{Peru}
	      {Title: Electrical resistivity tomography applied to the study of roots growth.}
\cventry{2006}{Bachelor of science with mention in Electronic Engineering}
	      {UNI}
	      {}{Peru}{}

\subsection{Areas of expertise}

\cvitem{}{Electronic engineering, information theory, error correcting codes, programing, electronic design, digital signal processing.}

%\section{Masters Thesis}
%
%\cvitem{Title}{\emph{Money Is The Root Of All Evil -- Or Is It?}}
%\cvitem{Supervisors}{Professor James Smith \& Associate Professor Jane Smith}
%\cvitem{Description}{This thesis explored the idea that money has been the cause of untold anguish and suffering in the world. I found that it has, in fact, not.}

%----------------------------------------------------------------------------------------
%	WORK EXPERIENCE SECTION
%----------------------------------------------------------------------------------------

\section{Experience}

\subsection{Teaching experience}

%\cventry{2012--Present}{}{}{}{}	{ \newline{} }
\cventry{Second semester 2018}{PSI528 - Signal processing}
	      {Engineering Department}
	      {UFLA}{Brazil}
	      {30 hours }
\cventry{First semester 2018}{PSI528 - Signal processing}
	      {Engineering Department}
	      {UFLA}{Brazil}
	      {30 hours }

\cventry{November 2016}{short course: Dynamic Speckle Laser in Bio-systems}
	      {Entity: Faculty of Agricultural Engineering}
	      {UNICAMP}{Brazil}
	      {8 hours }

\cventry{Second semester 2013}{Teacher training stage: PED C}
			  {GL100 }{Mathematics I}{}
			  {Entity: FCA UNICAMP }

\cventry{First semester 2010}{Teacher training stage: PED C}
			  {EE881 }{Communications principles}{}
			  {Entity: FEEC UNICAMP }

\cventry{2008}{Teacher}
	      {C++ Language}{ Level I}{}
	      {Entity: CCIESAM - UNI. Peru.}

\subsection{Professional experience}
\cventry{2015 -- 2020}{Postdoctoral}
	      {University of Lavras (UFLA)}
	      {}{Brazil}
	      {Engineering department / Applied Instrumentation Development Center to Agriculture (CEDIA) }


\cventry{2007 -- 2008}{Researcher}
		      {Institute for Research and Development of Civil Engineering Faculty (IIFIC)}
		      {UNI}{Peru}
		      {Type of contract: Labor\newline{}
		      Description: Design, construction and data processing of an accelerometer 
		      to the Accelerometers National Network  of CISMID - II.}

\cventry{2006 -- 2008}{Researcher}
		      {IIFIC}{UNI }{Peru}
		      {Type of contract: Labor\newline{}
		      Description: Design and construction  of  a data acquisition system for dynamic testing of piles.}

%\cventry{2005 -- 2006}{Researcher}
%		      {Center  of Research and Development of the Faculty of Electrical Engineering and Electronics (CID-FIEE)}
%		      {UNI}{Peru}
%		      {Type of contract: Labor\newline{}
%		      Description: Study, evaluation, design and implementation of a bioelectronic system II.}

%\cventry{2004 -- 2006}{Researcher}
%		      {International Potato Center}{Lima}{Peru}
%		      {Type of contract: Scholarship researcher\newline{}
%		      Description: Construction of an electrical resistivity tomograph applied to study of roots growth.}

%----------------------------------------------------------------------------------------
%	AWARDS SECTION
%----------------------------------------------------------------------------------------
%\section{Awards}
%\cvitem{2011}{School of Business Postgraduate Scholarship}
%\cvitem{2010}{Top Achiever Award -- Commerce}


\section{Published works}
\subsection{Books}
\cventry{2020}{Métodos numéricos: Problemas não lineares e inversos}
	      {ISBN: 978-65-00-07314-0}{2020}{Edição independente}
	      {https://trucomanx.github.io/metodos.numericos/index.html}
\cventry{2016}{A practical guide to biospeckle laser analysis: theory and software}
	      {ISBN: 978-85-81-27051-7}{2016}{Ed. UFLA}
	      {http://repositorio.ufla.br/jspui/handle/1/12119}

\subsection{Chapters of Books}
\cventry{2019}{Engenharias, ciência e tecnologia 4}
	      {ISBN: 978-85-72-47087-2}{2019}{Editora Atena}
	      {DOI:10.22533/at.ed.87219310127}

\subsection{Articles published in magazines}

\cventry{2020}{Brazilian Journal of Development}
	      {DOI: 10.34117/bjdv6n5-072}{}{}
	      {Title: ``Use of particle image velocimetry (PIV) to study the modulus of elasticity of plywood panels''.}

\cventry{2020}{Brazilian Journal of Development}
	      {DOI: 10.34117/bjdv6n5-069}{}{}
	      {Title: ``Use of the velocimetry technique by particle images (PIV) for the study of deformations in pinus oocarpa wood panels''.}

\cventry{2020}{Brazilian Journal of Development}
	      {DOI: 10.34117/bjdv6n5-074}{}{}
	      {Title: ``Use of the Particle Imaging Velocimetry (PIV) technique to obtain the deformation map in Pinus Oocarpa wood panels''.}

\cventry{2020}{Optics And Laser Technology}
	      {DOI: 10.1016/j.optlastec.2020.106221}{}{}
	      {Title: ``Illumination dependency in dynamic laser speckle analysis''.}

\cventry{2019}{Computers and Electronics in Agriculture}
	      {DOI: 10.1016/j.compag.2019.105050}{}{}
	      {Title: ``Development of an optical technique for characterizing presence of soil surface crusts''.}

\cventry{2019}{CERNE}
	      {DOI: 10.1590/01047760201925022633}{}{}
	      {Title: ``Particle image velocimetry for estimating the young’s modulus of wood specimens''.}

\cventry{2019}{Optik}
	      {DOI: 10.1016/j.ijleo.2019.02.055}{}{}
	      {Title: ``Viability of biospeckle laser in mobile devices''.}

\cventry{2019}{CERNE}
	      {DOI: 10.1590/01047760201925012619}{}{}
	      {Title: ``Displacement measurement in sawn wood and wood panel beams using particle image velocimetry''.}

\cventry{2019}{Computers and Electronics in Agriculture}
	      {DOI: 10.1016/j.compag.2019.01.051}{}{}
	      {Title: ``Sound as a qualitative index of speckle laser to monitor biological systems''.}

\cventry{2018}{Theoretical and Applied Engineering}
	      {DOI: 10.31422/taae.v2i2.5}{}{}
	      {Title: ``The use of particle image velocimetry for displacement measurements in steel columns subjected to buckling''.}
	      
\cventry{2018}{Optics and Laser Technology}
	      {DOI: 10.1016/j.optlastec.2018.07.006}{}{}
	      {Title: ``Diode laser reliability in dynamic laser speckle application: Stability and signal to noise ratio''.}
	      
\cventry{2018}{Journal of Food Measurement and Characterization}
	      {DOI: 10.1007/s11694-018-9839-8}{}{}
	      {Title: ``Measurement of water activities of foods at different temperatures using biospeckle laser''.}

\cventry{2018}{Engenharia Agrícola, ISSN:0100-6916}
	      {DOI: 10.1590/1809-4430-eng.agric.v38n2p159-165/2018}{}{}
	      {Title: ``Analysis of elasticity in woods submitted to the static bending test using the particle image velocimetry (PIV) technique''.}

\cventry{2017}{Journal of Biomedical Optics}
	      {DOI: 10.1117/1.JBO.22.4.045010}{}{}
	      {Title: Dynamic laser speckle analyzed considering inhomogeneities in the biological sample.}
	      
\cventry{2017}{Optics Communications}
	      {DOI: 10.1016/j.optcom.2017.03.015}{}{}
	      {Title: Selection of statistical indices in the biospeckle laser analysis regarding filtering actions.}
	      
\cventry{2014}{IEEE Communications Letters}
	      {DOI: 10.1109/LCOMM.2014.2377237}{}{}
	      {Title: Optimal  Rate for Joint Source-Channel Coding of Correlated Sources Over Orthogonal Channels.}

\subsection{Articles published in annals of events}
\cventry{2015}{I Congresso Mineiro de Engenharia e Tecnologia}
	      {Brasil}{}{http://www.eventos.ufla.br/comet/ANAIS\_COMET\_2015\_1ed\_FINAL.pdf}
	      {Title:``Diferenciação da Crosta Superficial do Solo por Meio de Técnicas Óticas''}
\cventry{2013}{XXXI Brazilian Telecommunications Symposium}
	      {Brazil}{}{}
	      {Title: ``Algoritmo Para Decodificação e Fusão De Dados Correlacionados Em Redes De Sensores Sem Fio''.}

\cventry{2012}{XXX Brazilian Telecommunications Symposium}
	      {Brazil}{}{}
	      {Title: ``Algoritmos de Decodificação Abrupta para Códigos LDGM''.}

\cventry{2011}{XXIX Brazilian Telecommunications Symposium}
	      {Brazil}{}{}
	      {Title: ``Decodificação Iterativa Conjunta Fonte-Canal''.}

%\cventry{2007}{XVII  National Congress of Engineering, Mechanical, Electrical and Allied}
%	      {Peru}{}{}
%	      {Title: ``Tomógrafo de Resistividad Eléctrica Aplicado al Estudio del Crecimiento de los Tubérculos de la Papa''.}


	       
%----------------------------------------------------------------------------------------
%	COMPUTER SKILLS SECTION
%----------------------------------------------------------------------------------------
\section{Professor adviser}
\subsection{Joint supervisor}
\cventry{2017}{Study of trajectories reconstruction based on low cost inertial sensors and applied to terrestrial mobility context}
			{Ribeiro, Eduardo Zampieri}
			{Master's degree in Systems Engineering and Automation}{UFLA}
			{http://repositorio.ufla.br/handle/1/28225}
\cventry{2016}{Development of an optic technique for characterizing the presence of superficial crust of the soil}
			{Barreto, Bianca Batista}
			{Master's degree in Agricultural Engineering}{UFLA}
			{http://repositorio.ufla.br/jspui/handle/1/11903}


\section{Participation in stalls completion work}

\subsection{Doctoral's degree}
\cventry{2016}{Digitization of physical deformations of the soil through a digital camera}
			{Participation in stalls of Diego Eduardo Costa Coelho}
			{Dissertation defense of post-graduation program agricultural engineering}{}
			{UFLA. Ordinance CPGSS/PRPG Nro 987/2016 de 23/11/2016.}
			
\subsection{Master's degree}
\cventry{2017}{Low cost inertial sensor-based trajectory generation: Application in intelligent transport systems}
			{Chairman of the stall of Eduardo Zampieri Ribeiro}
			{Dissertation defense of post-graduation program  in system and automation engineering }{}
			{UFLA. Ordinance CPGSS/PRPG Nro 563/2017 de 11/10/2017.}
\cventry{2015}{Influence of laser intensity in the biospeckle actvity map}
			{Participation in stalls of Renan Oliveira Reis}
			{Dissertation defense of post-graduation program  in system and automation engineering }{}
			{UFLA. Ordinance CPGSS/PRPG Nro 655/2015 of 13/07/2015.}

\subsection{Doctoral's degree qualification}
\cventry{2019}{Participation in the evaluation committee of Elisângela Ribeiro}
			{}
			{Qualification exam of post-graduation program  in agricultural engineering}{}
			{Universidade Federal de Lavras.}
\cventry{2019}{Participation in the evaluation committee of Bianca Batista Barreto}
			{}
			{Qualification exam of post-graduation program  in agricultural engineering}{}
			{Universidade Federal de Lavras.}
\cventry{2016}{Participation in the evaluation committee of Rodrigo Allan Pereira}
			{}
			{Qualification exam of post-graduation program  in agricultural engineering}{}
			{UFLA. }

\subsection{Master's Degree Qualification}
\cventry{2018}{Participation in the evaluation committee of  Thiago Juvenal Ribeiro}
			{}
			{Qualification exam of post-graduation program  in agricultural engineering}{}
			{UFLA.}
\cventry{2018}{Participation in the evaluation committee of  Dione Weverton Dos Reis Araújo}
			{}
			{Qualification exam of post-graduation program  in system and automation engineering}{}
			{UFLA.}
\cventry{2016}{Participation in the evaluation committee of Eduardo Zampieri Ribeiro}
			{}
			{Qualification exam of post-graduation program  in system and automation engineering}{}
			{UFLA. }



\section{Complementary Training}

\subsection{Complementary Training Courses}
\cventry{2020}{Object detection}
	      {6 weeks}
	      {\url{http://coursera.org/verify/FQA75P2H8JLS}}{}
	      {an online non-credit course authorized by Universitat Autònoma de Barcelona and offered through Coursera.}
\cventry{2020}{Machine Learning}
	      {11 weeks}
	      {\url{http://coursera.org/verify/TLNHXEJP22ZB}}{}
	      {an online non-credit course authorized by Stanford University and offered through Coursera.}
\cventry{2020}{Machine Learning for All}
	      {20 Horas}
	      {\url{http://coursera.org/verify/CZE8NBUCW87H}}{}
	      {An online non-credit course authorized by University of London and offered through Coursera.}

%\cventry{2008}{Automatização por PLC}
%	      {30 Horas}
%	      {}{}
%	      {Centro Cultural de Engenharia Elétrica ``Santiago Antúnez de Mayolo'' 
%	      (CCIESAM) da Faculdade de Engenharia Elétrica e Eletrônica da Universidade 
%	      Nacional de Engenharia, Peru.}

%\cventry{2007}{Compatibilidade Electromagnética}
%	      {9 Horas}
%	      {}{}
%	      {Secção de Pós-Graduação e Segunda Especialização da Faculdade de 
%	      Engenharia Elétrica e Eletrônica da Universidade Nacional de 
%	      Engenharia, Peru.}


%\cventry{2004}{Ensamblagem de Computadoras}
%	      {20 horas}
%	      {}{}
%	      {Centro de Capacitação Administrativa da Universidade Nacional de 
%	      Engenharia, a través do Centro de Computo ARUNI, Peru.}

%\cventry{2003}{Borland C++ 5.5 Nivel II}
%	      {20 horas}
%	      {}{}
%	      {Centro de Extensão e Projeção Social da Universidade Nacional de Engenharia.}

%\cventry{2003}{Programação em PHP}
%	      {24 horas}
%	      {}{}
%	      {Rama Estudantil IEEE da Universidade Nacional de Engenharia, Peru.}

%\cventry{2001}{Borland C++ Nivel I}
%	      {25 horas}
%	      {}{}
%	      {Centro de Especialização em Rede e Telemática da Faculdade de 
%	      Engenharia Elétrica e Eletrônica da Universidade Nacional de 
%	      Engenharia, Peru.}

%----------------------------------------------------------------------------------------
%	COMPUTER SKILLS SECTION
%----------------------------------------------------------------------------------------



%----------------------------------------------------------------------------------------
%	COMMUNICATION SKILLS SECTION
%----------------------------------------------------------------------------------------

\section{Presentations}

\cventry{2013}{Algorithm for decoding and fusion of correlated data  in wireless sensor networks}
	      {}{}{}
	      {XXXI Brazilian Telecommunications Symposium, Brazil}

\cventry{2012}{Hard-decision decoding algorithms for LDGM codes}
	      {}{}{}
	      {XXX Brazilian Telecommunications Symposium, Brazil}

\cventry{2011}{Iterative source-channel joint decoding}
	      {}{}{}
	      {XXIX Brazilian Telecommunications Symposium, Brazil}

%\cventry{2008}{Free tools in CAD design, KiCad}
%	      {}{}{}
%	      {``I Jornada de Software Libre GNU/LINUX 2008 FIEE-UNI'', Peru}

%\cventry{2008}{O Compilador PIC-GCC e as Bibliotecas PIC-GCC-Library}
%	      {}{}{}
%	      {``I Jornada de Software Libre GNU/LINUX 2008 FIEE-UNI'', Peru}

%\cventry{2007}{``Tomógrafo de Resistividad Eléctrica Aplicado al Estudio del Crecimiento de los Tubérculos de la Papa''}
%	      {}{}{}
%	      {``XVII  Congreso Nacional de Ingeniería, Mecánica, Eléctrica y Ramas Afines'', Peru}

%\cvitem{2006}{Expositor dos Projetos da Área de Processamento de Sinais e Sistemas 
%	      do Centro de Investigação e Desenvolvimento da Faculdade de Engenharia
%	      Elétrica e Eletrônica, ``Feria Científica Tecnológica y Empresarial UNI 2006'', Peru}


%----------------------------------------------------------------------------------------
%	LANGUAGES SECTION
%----------------------------------------------------------------------------------------

\section{Languages}

%\cvitemwithcomment{Espanhol}{Língua Materna}{}
\cvitem{Spanish}{Native language}
\cvitem{Portuguese}{Read good, write good, understands good, speak good}
\cvitem{English}{Read good, Write reasonably, Understands reasonably, Speaks little}

 
%----------------------------------------------------------------------------------------
%	COMPUTER 
%----------------------------------------------------------------------------------------
\section{Free software projects}

\cventry{2015 -- Actual}{Bio-Speckle Laser Tool Library}
			{\url{http://www.nongnu.org/bsltl/}}
			{}{}
			{This package is a set of functions, written in M-code, 
			for the digital processing of images  of a bio-speckle analysis.
			The library is designed to be used in OCTAVE or MATLAB.
			You can find functions to calculate:
			Co-occurrence matrix, THSP, AVD, inertia moment,
			Fujii, GD, PTD, etc.}

\cventry{2015 -- Actual}{PDS-IT Package}
			{\url{http://trucomanx.github.io/pdsit-pkg}}
			{}{}
			{This package is a set of functions, written in M-code, for to work
			with digital signal processing and information theory
			in OCTAVE or MATLAB. You can find functions for:
			Entropy for binary sources, Joint entropy for binary sources,
			bit error rate in the CEO problem, etc. }
			
\cventry{2014 -- Actual}{PDS Project Library in Java}
			{\url{http://pdsplibj.sourceforge.net/}}
			{}{}
			{It is a set of libraries, written in Java language, For
			the digital signal processing. You can find
			libraries for: Random variables,
			vectors, matrices, digital filters,
			digital sources, particle image velocimetry, etc.}

\cventry{2014 -- Actual}{LDPC Tools}
			{\url{https://launchpad.net/ldpc-tools}}
			{}{}
			{It is a set of programs, written in C language, for
			to work with low density parity check matrices.}
			

\cventry{2011 -- Actual}{PDS Project Library}
			{\url{http://www.nongnu.org/pdsplibrary/}}
			{}{}
			{It is a set of libraries, written in C language, for 
			the digital signal processing. You can find libraries for:
			Random variables, complexs numbers, vectors, matrices, FFT,
			digital filters, digital sources, neural networks, etc.}

\cventry{2008 -- Actual}{PIC-GCC Library}
			{\url{http://pic-gcc-library.sourceforge.net/}}
			{}{}
			{This project implement the utility library and 
			standard C library for the PIC-GCC compiler for 
			micro-controllers PIC of Microchip 16F family. }


\section{Computer languages}

\cvitem{C}{C language }
\cvitem{M-code}{MATLAB/OCTAVE language }
\cvitem{C++}{C++ language }
\cvitem{Java}{Java language }
\cvitem{LaTeX}{LaTex language }
\cvitem{Java/Android}{Development of Android applications - Basic level}

%----------------------------------------------------------------------------------------
%	INTERESTS SECTION
%----------------------------------------------------------------------------------------

\section{Interests}

\renewcommand{\listitemsymbol}{-~} % Changes the symbol used for lists
\cvlistdoubleitem{Photography}{Running}
\cvlistdoubleitem{Ocarinas maker}{C language}
\cvlistdoubleitem{Dance}{Raw food}
%\cvlistdoubleitem{}{}

\end{document}
